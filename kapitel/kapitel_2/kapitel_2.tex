\newpage
\section{Modelle}
\subsection{Agiles Modell}
Schon seit den frühen 90er Jahren wurden Trigger in einigen Produkten unterstützt, allerdings erst im SQL-99 Standard aufgenommen\footcite[Vgl.][2]{shao2006triggers}. Trigger sind benamte PL/SQL Objekte, welche in der Datenbank hinterlegt werden und beliebig oft ausgeführt werden können. Ein Trigger lässt sich aktivieren bzw. deaktivieren, allerdings nicht explizit ausführen. Solange er aktiv ist, wird er von der Datenbank automatisch ausgeführt bzw. gefeuert, sobald das für diesen Trigger definierte Event eintritt.\footcite[Vgl.][S. 9-1 f.]{oracle}

Die Events, welche benutzerdefinierte Trigger auslösen, sind laut dem Industriestandard DML Befehle. In einer relationalen Datenbank können Trigger zum Beispiel dergestalt definiert werden, dass sie dann gefeuert werden, wenn eine Zeile einer Tabelle oder eines Views geändert, eingefügt oder gelöscht wird. Dementsprechend ist jeder Trigger einer Tabelle der Datenbank zugeordnet. Das bedeutet, dass die Reichweite von benutzerdefinierten Triggern in einem konventionellen DBMS der Tabellenebene der Datenbank entspricht.\footcite[Vgl.][]{samu2002database} In so einem Fall spricht man von einem "`DML Trigger"'. Ist ein Trigger für ein Schema oder die Datenbank selbst angelegt, ist das Event aus DDL oder Datenbankoperationsbefehlen zusammengesetzt und wird "`System Trigger"' genannt.\footcite[Vgl.][9-2]{oracle}





\subsection{Anderes Modell}
Das folgende Kapitel basiert, falls nicht explizit anders erwähnt, auf den Ausführungen von Alpern et al. (2016), Seite 9-1 ff. Zur Referenz wird am Ende eines jeden Abschnitts auf die entsprechende Stelle der Quelle verwiesen.




\subsection{Noch ein anderes Modell}
Die Sprache zur Spezifikation eines Triggers hat die folgende Syntax:

\lstinputlisting[language=SQL]{./Quellcode/triggerdef.sql}

