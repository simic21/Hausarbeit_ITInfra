\newpage
\section{Schlussbetrachtung}
\subsection{Fazit}
Ein Smart Workplace ist die Ausrichtung der Prozesse und Policies eines Unternehmens auf verschiedene Komponenten wie flexibel einsetzbare Endgeräte oder die moderne Cloud- und Virtualisierungstechnologie. Dementsprechend ist es eine interessante und aktuelle Implementierungsmöglichkeit für Ambient Intelligence Applikationen. Sie umfasst dabei allerdings nicht nur die typischen Technologien wie Ubiquitous Computing, sondern auch die zuvor genannten Unternehmenspolicies sowie die Arbeitsmethodik und den physikalischen Aufbau des Unternehmens.

Bezüglich technischen Aspekte, die in einem Smart Workplace eines mittelständischen Unternehmens implementiert werden können, wurden die MOVE-Architektur, die CGLXTouch-Technologie und das Mobile Cloud System aufgeführt und erläutert. Die MOVE-Architektur dient als persönliche mobile Arbeitsumgebung und stellt dem Benutzer an unterschiedlichen Clients des Unternehmens denselben, eigenen Arbeitsplatz zur Verfügung. Dies setzt lediglich die Einrichtung der einzelnen Clients, eines MOVE-Servers zur Verwaltung der individuellen Arbeitsplätze und eines verbindenden IP-Netzwerks voraus, welches zumeist bereits vorhanden ist. Daher ist es eine sinnvolle und umsetzbare Architektur für ein mittelständisches Unternehmen.

Auch die CGLXTouch-Technologie, bestehend aus einer geteilten Displayumgebung, einem Head Node und Render Nodes, lässt sich praktikabel umsetzen. Steuerungselemente, wie der Multi-Touch-Table und die Mobilgeräte, sind optional, zumal heutzutage jeder Mitarbeiter zumindest ein Smartphone bei sich trägt und somit jederzeit eine Verbindung zu dem System herstellen kann.
Damit lassen sich Meetings in Zukunft interaktiver gestalten.

Je nach Art der Implementierung kann ein Mobile Cloud System so aufgesetzt werden, dass ein Remote Server die angefallenen Aufgaben übernimmt, oder die Aufgaben werden an alle verfügbaren Mobilgeräte im Unternehmen verteilt. Beide Varianten sind in jedem Fall umsetzbar, da lediglich ein Remote Server benötigt wird, auf dem die Verwaltungssoftware eingerichtet wird.
Wearables lassen sich in dieses System unter der Verwendung der Energiesparmethoden ebenfalls einbinden.

Alle aufgeführten Technologien lassen sich mit den rechtlichen Bestimmungen vereinbaren, sofern die entsprechenden Anforderungen der Arbeitsplatzergonomie erfüllt werden, wie beispielsweise die Verwendung einer externen Tastatur für den Laptop oder die Sicherstellung einer gesunden Arbeitsumgebung.

\subsection{Ausblick}
In Zukunft könnten Wearables expliziter in das Smart Workplace eingebunden werden, wenn sowohl die technologische Entwicklung der Wearables selbst voranschreitet als auch die Restriktion durch die Abhängigkeit vom Smartphone entfällt. Ebenfalls könnten weitere Sensoren implementiert werden, die, wie bereits erwähnt wurde, die Umgebung und Verhaltensweisen der Mitarbeiter erfassen, wodurch Kosteneinsparungen erzielt werden können. Hierbei gilt es allerdings zu prüfen, ob solch eine Überwachung gesetzlich erlaubt ist oder bereits unter Mitarbeiterüberwachung fällt. In ferner Zukunft wäre es denkbar, dass Meetings durch den Einsatz von beispielsweise gestengesteuerten holographischen Projektionen noch interaktiver gestaltet werden können. Dies ist zurzeit allerdings lediglich eine Vorstellung.