\section{Einleitung}
\subsection{Themenvorstellung}
In der heutigen Zeit tendieren Unternehmen vermehrt dazu, zur Vereinfachung und Prozessoptimierung neue Technologien in den Arbeitsalltag einfließen zu lassen. Dies hat ebenfalls einen Einfluss auf die Arbeitsumgebungen der Mitarbeiter. Die standardmäßigen Büros mit Arbeitsplatzrechnern weichen den Großräumen mit mobilen Endgeräten, Informationen wie E-Mails können über zahlreiche Medien erhalten werden und Meetings werden interaktiv und spontan an verschiendenen Orten gehalten. Zuletzt hat das Unternehmen Microsoft in seinem münchener Standort so eine intelligente Arbeitsumgebung, auch Smart Workplace oder Smart Workspace genannt, eingerichtet.\footcite[Vgl.][]{MicrosoftArtikel} Diese moderne Entwicklung ist die Motivation hinter dieser Seminararbeit.

\subsection{Zielsetzung}
Im Folgenden steht die Darstellung eines Smart Workplace und dessen mögliche Umsetzungen im Vordergrund. Es sollen Anwendungsfälle für ein mittelständisches Unternehmen und die dafür notwendigen Voraussetzungen vorgestellt werden. Auch eventuelle Konflikte mit aktuellen gesetzlichen Verordnungen bezüglich der Arbeitspaltzergonomie werden eruiert und versucht mit dem Einsatz neuer Technologien in Einklang zu bringen.

\subsection{Aufbau}
Beginnend wird der Begriff des Smart Workplace näher definiert und erläutert. Anschließend werden die einsetzbaren Technologien und räumlichen Gestaltungsmöglichkeiten dargestellt und gleichzeitig die Voraussetzungen für deren Umsetzung gepüft. Die Einführung der Systeme wird schrittweise dargelegt. Sollten bezüglich des Einsatzes der Technologien oder der Raumgestaltung Konflikte mit den rechtlichen Grundlagen existieren, werden diese aufgezeigt und mögliche Gesetzesanpassungen präsentiert um die Nutzung zukünfitig zu ermöglichen. Abschließend wird die Arbeit in einem Fazit reflektiert und die wesentlichen inhaltlichen Aspekte zusammengefasst.