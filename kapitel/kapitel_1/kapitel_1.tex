\newpage
\section{Smart Workplace}
\subsection{Begirffsdefinition}
Das erste Hindernis stellt eine genauere Definition und Erläuterung des Begriffs Smart Workplace dar. 

Bei einem Smart Workpalce werden die Prozesse und Policies eines Unternehmens auf verschiedene Komponenten ausgerichtet. Diese bestehen aus flexibel einsetzbaren Endgeräten, die mit neuen Technologien wie Cloud oder Virtualsierung angebunden und untereinander intelligent vernetzt werden um die notwendigen Informationen verfügbar zu machen.\footcite[Vgl.][]{nextchange}

Beispielsweise können individuelle Licht- und Temperatureinstellungen automatisiert vom Gebäude an spezifische Büros vorgenommen werden, Sensoren in Konferenzräumen können Benutzungstrends auswerten und diese in Outlook oder Google Kalendern zur dynamischen Raumplanung ausgeben oder Ähnliches. Zusammengefasst können diese scheinbar kleinen Verbesserungen Ablenkungen vermindern und die Konzentration der Mitarbeiter erhöhen.\footcite[Vgl.][]{iotagenda}

Die Einführung und Implementation des Smart Workplace in einem Unternehmen umfasst vier Bereiche:

\begin{itemize}
\item[1.] Digitale Technologien
\item[2.] Unternehmenspolicies
\item[3.] Führungsstile und Verhalten im Unternehmen
\item[4.] Physikalischer Aufbau\footcite[Vgl.][]{efm}
\end{itemize}

Digitale Innovationen haben unseren Arbeitsalltag stark beeinflusst und dadurch sowohl die Leistungs- als auch die Wettbewerbsfähigkeit von Unternehmen verbessert. Die Technologien, die heute für Smart Working verwendet werden unterstützen die Zusammenarbeit und Sozialisierung und bieten Zugang zu Informationen außerhalb des Unternehmens. Die Arbeitsbedingungen der Unternehmen tendieren immer mehr zu flexibleren Modellen, die Verringerung beziehungsweise Entfernung der Einschränkungen durch Zeit und Räumlichkeiten ist ein entscheidender Erfolgsfaktor im Einführungsprozess des Smart Working.\footcite[Vgl.][]{efm}

Wenn sich die Arbeitsmethodik der Menschen ändert, muss sich zur effektiveren Unterstützung der Mitarbeiter auch das Büro weiterentwickeln. Bei der Konzeptionierung eines Smart Office geht es nicht nur darum, die Anzahl der Workstations zu reduzieren um das Nutzungsniveau zu erhöhen, sondern generell die Bedeutung und Logik der Arbeitsbereiche zu überdenken. Dabei lassen sich vier Bedürfnisse hinter den Arbeitsaktivitäten feststellen:

\begin{itemize}
\item Konzentration: eine ruhige Umgebung abseits von lärmenden Orten
\item Kollaboration: Räume mit genügen Platz zum Austausch von Dokumenten und ausgestattet mit adequaten und fexiblen technologischen Elementen um eine  Zusammenarbeit per Remote zu ermöglichen
\item Kommunikation: Schalldämmung und ein hohes Maß an Vertraulichkeit für den Umgang mit vertraulichen Angelegenheiten und Technologien zur Mischung physischer und virtueller Kommunikation
\item Reflexion: Umgebungen für Arbeitsunterbrechungen mit der Möglichkeit individuelles kreatives Denken durchzuführen\footcite[Vgl.][]{efm}
\end{itemize}

\subsection{Ambient Intelligence}
Structured Query Language (SQL) \nomenclature{SQL}{Structured Query Language}