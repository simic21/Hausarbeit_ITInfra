\newpage
\section{Smart Workplace}
\subsection{Definition}


\subsection{Zuständigkeitsbestimmung}
Structured Query Language (SQL) \nomenclature{SQL}{Structured Query Language}ist eine nicht-prozedurale Sprache, welche in den meisten relationalen Datenbanken geläufig ist. Sie wird sowohl von der Datenbank zur internen Selbstverwaltung als auch von Benutzern verwendet, um die dort gespeicherten Daten zu manipulieren und abzufragen. Außerdem lässt sich mit SQL Befehlen zum Beispiel zwischen verschiedenen Datenbanksystemen bewegen und Abfragen generieren welche in der Benutzeroberfläche so nicht repräsentiert werden oder zur Performanceanalyse genutzt werden können.\footcite[Vgl.][]{UniversityofDelaware}

Es handelt sich hierbei um eine dem Englischen ähnliche Sprache. Sie wurde 1974 von der International Business Machines Corp. (IBM) \nomenclature{IBM}{International Business Machines}erfunden, wird stetig erweitert und ist mittlerweile de-facto weltweiter Standard. Bestehend aus rund 60 Befehlen lässt sie sich auf fast jedem Computersystem ausführen.\footcite[Vgl.][]{BusinessDictionary}
Die am häufigsten anzutreffenden Befehle sind SELECT, INSERT, UPDATE, CREATE, UNION SELECT und DELETE\footcite[Vgl.][Seite 486]{OlearySteele}.
Diese Befehle gehören zur Data Definition Language (DDL)\nomenclature{DDL}{Data Definition Language}, welche zur Definition oder Anpassung von Datenbankstrukturen genutzt wird. Befehle, welche Daten selbst selektieren oder ändern wie INSERT, UPDATE oder DELETE gehören der Data Manipulation Language (DML) \nomenclature{DML}{Data Manipulation Language}an.\footcite[Vgl.][]{UniversityofDelaware}

\subsection{Teambildung}
Procedural Language / Structured Query Language (PL/SQL)\nomenclature{PL/SQL}{Procedural Language / Structured Query Language}, die prozedurale Erweiterung von SQL durch Oracle, ist eine portable, high-performance Transaktionsverarbeitungssprache. Sie kombiniert die Datenmanipulationsfunktion von SQL mit der Verarbeitungsfunktion von prozeduralen Sprachen. In PL/SQL ist es wie in anderen prozeduralen Programmiersprachen ebenfalls möglich, Konstanten und Variablen zu deklarieren, Laufzeitfehler aufzudecken und Unterprogramme zu definieren.\footcite[Vgl.][Seite 1-1 ff]{oracle}

\subsection{Aufgabenverteilung}
Da in dieser Arbeit nicht nur auf den allgemeinen Aufbau von Triggern, sondern auch auf die Spezialitäten auf Oracle Systemen eingegangen wird, gehört dies ebenfalls zu den Grundlagen.

Die Oracle Corporation ist ein IT-Dienstleistungsunternehmen, welches Datenbank- und Middleware-Softwarelösungen, Applikationssoftware und Computerhardware wie Server und Speichersysteme entwickelt, produziert, vermarktet und vertreibt. In diesem Bereich gehört Oracle zu den weltweit führenden Anbietern.\footcite[Vgl.][]{OracleCorp}

\subsection{Entwicklung}

\subsection{Qualitätssicherung}

\subsection{Abnahme}
