\newpage
\section{Smart Workplace}
\subsection{Begirffsdefinition}
Das erste Hindernis stellt eine genauere Definition und Erläuterung des Begriffs Smart Workplace dar. 

Bei einem Smart Workpalce werden die Prozesse und Policies eines Unternehmens auf verschiedene Komponenten ausgerichtet. Diese bestehen aus flexibel einsetzbaren Endgeräten, die mit neuen Technologien wie Cloud oder Virtualsierung angebunden und untereinander intelligent vernetzt werden um die notwendigen Informationen verfügbar zu machen.\footcite[Vgl.][]{nextchange}

Beispielsweise können individuelle Licht- und Temperatureinstellungen automatisiert vom Gebäude an spezifische Büros vorgenommen werden, Sensoren in Konferenzräumen können Benutzungstrends auswerten und diese in Outlook oder Google Kalendern zur dynamischen Raumplanung ausgeben oder Ähnliches. Zusammengefasst können diese scheinbar kleinen Verbesserungen Ablenkungen vermindern und die Konzentration der Mitarbeiter erhöhen.\footcite[Vgl.][]{iotagenda}

Dementsprechend lässt sich sagen, dass Smart Workplaces eine Implementierungsmöglichkeit für Ambient Intelligence (AmI)\nomenclature{AmI}{Ambient Intelligence} Applikationen darstellen.\footcite[Vgl.][Seite 401]{leseprobe}

\subsection{Ambient Intelligence}
Intelligente Arbeitsplätze gehören zu einem der interessantesten Bereiche für Ambient Intelligence Anwendungen. 
Hierbei werden intelligente Schnittstellen in alle möglichen Arten von Objekten eingebettet und erzeugen eine Umgebung, die anwesende Personen erkennen und auf diese in einer nahtlosen und unauffälligen Art und Weise zu reagieren.\footcite[Vgl.][Seite 1]{istag} Diese Anwendungen sollen standardmäßige Arbeitsplätze durch solche Objekte der Informations- und Kommunikationstechnologie verbessern, indem diese in der Umgebung verborgen und von Nutzern auf intuitive Weise zur Gewinnung von Informationen und zur Problemlösung genutzt werden.\footcite[Vgl.][Seite 401]{leseprobe}

AmI Architekturen werden unter anderem durch einige Schlüsselfunktionen charakterisiert. Hierzu gehört zum einen die Einbettung der Geräte ins Netwerk, kabelgebunden oder kabellos. Diese Geräte reichen von einfachen Sensoren bis hin zu Kontrollsystemen wie Sicherheitssystemen und sollen in der Lage sein miteinander zu interagieren, was sich allerdings aufgrund der Heterogenität der Systeme oft als schwierig herausstellt. Zum anderen müssen AmI Architekturen die Funktionalität besitzen, Personen in der Umgebung und Situationszusammenhänge zu erkennen.\footcite[Vgl.][Seite 585]{SpecialIssue}

Weiterhin sollten AmI Umgebungen möglichst benutzerfreundlich gestaltet sein, da die User solcher Systeme zumeist alle Mitarbeiter sind. Daher sollten sie beispielsweise Schnittstellen für Spracherkennung und anderen Sensoren bestizen. Abschließend ist die Anpassungsfähigkeit der Systeme an wechselnde Umstände von Bedeutung. Sie müssen gemäß den Rückmeldungen der Nutzer ihre Aktionen anpassen, dabei allerdings das Leistungsniveau beibehalten. Diese Vorgaben haben AmI Systeme im Bezug auf deren Verhalten bei Interaktionen vieler Geräte mit der Zeit so komplex gemacht wie natürliche Systeme.\footcite[Vgl.][Seite 586]{SpecialIssue}

Spricht man von AmI Infrastrukturen, fallen stets drei aktuelle Technologien als Kernaspekte. Die erste Technologie nennt sich Ubiqutious Computing.TODOAmIBlaBla Diese Technologie wurde in dieser Hausarbeit bereits oft aufgeführt, und zwar betrifft sie Systeme, welche aus dem Sichtfeld der Nutzer entfernt werden und nur im Hintergrund agieren. Sie dienen dazu, Informationen zu akkumulieren und diese in der physischen Welt zu integrieren und darzustellen. Dadurch sind diese Informationen überall und für jede Person zugänglich. Die zur Anzeige der Informationen genutzten Displays sollen hierbei physische Daten wie Wandplakate, SMS oder Post-its ersetzen und diese in die Arbeitsumgebung integrieren.TODOUbiq-Weiser 

Der zweite Aspekt ist Ubiquitous Communication. Dies bezeichnet nichts anderes als die Funktionalität von Kommunikation mehrerer Geräte untereinander und mit Usern selbst unter Verwendung von kabellosen Technologien. Und zuletzt wird das Intelligent User Interface genannt. Den Benutzern soll ermöglicht werden, mit solch einer intelligenten Architektur sowohl auf eine natürliche Weise als auch auf eine auf den Nutzer personalisierte Art und Weise zu interagieren.TODOAmIBlaBla

Viele Applikationen von AmI Umgebungen treffen auf ähnliche Problemstellungen wie Identifikation von Personen, Energieverbrauch oder Data Mining um nur einige davon zu nennen. Einige dieser Problemstellungen lassen sich mit Soft Computing Ansätzen lösen.TODOAmiBlaBla TODOSoftComputing 

\subsection{Einführung}
Die Einführung und Implementation des Smart Workplace in einem Unternehmen umfasst vier Bereiche:

\begin{itemize}
\item[1.] Digitale Technologien
\item[2.] Unternehmenspolicies
\item[3.] Führungsstile und Verhalten im Unternehmen
\item[4.] Physikalischer Aufbau\footcite[Vgl.][]{efm}
\end{itemize}

Digitale Innovationen haben unseren Arbeitsalltag stark beeinflusst und dadurch sowohl die Leistungs- als auch die Wettbewerbsfähigkeit von Unternehmen verbessert. Die Technologien, die heute für Smart Working verwendet werden unterstützen die Zusammenarbeit und Sozialisierung und bieten Zugang zu Informationen außerhalb des Unternehmens. Die Arbeitsbedingungen der Unternehmen tendieren immer mehr zu flexibleren Modellen, die Verringerung beziehungsweise Entfernung der Einschränkungen durch Zeit und Räumlichkeiten ist ein entscheidender Erfolgsfaktor im Einführungsprozess des Smart Working.\footcite[Vgl.][]{efm}

Wenn sich die Arbeitsmethodik der Menschen ändert, muss sich zur effektiveren Unterstützung der Mitarbeiter auch das Büro weiterentwickeln. Bei der Konzeptionierung eines Smart Office geht es nicht nur darum, die Anzahl der Workstations zu reduzieren um das Nutzungsniveau zu erhöhen, sondern generell die Bedeutung und Logik der Arbeitsbereiche zu überdenken. Dabei lassen sich vier Bedürfnisse hinter den Arbeitsaktivitäten feststellen:

\begin{itemize}
\item Konzentration: eine ruhige Umgebung abseits von lärmenden Orten
\item Kollaboration: Räume mit genügen Platz zum Austausch von Dokumenten und ausgestattet mit adequaten und fexiblen technologischen Elementen um eine  Zusammenarbeit per Remote zu ermöglichen
\item Kommunikation: Schalldämmung und ein hohes Maß an Vertraulichkeit für den Umgang mit vertraulichen Angelegenheiten und Technologien zur Mischung physischer und virtueller Kommunikation
\item Reflexion: Umgebungen für Arbeitsunterbrechungen mit der Möglichkeit individuelles kreatives Denken durchzuführen.\footcite[Vgl.][]{efm}
\end{itemize}

Dem lässt sich entnehmen, dass nicht nur die technologische Ausstattung wichtig für ein Smart Workplace sind, sondern auch die räumliche Aufteilung des Gebäudes und sowie die Arbeitsbedingungen eine Rolle in der Einführung spielen.

Zudem existieren weitere Grundbausteine für die erfolgreiche Umsetzung eines Smart Workplace. Zunächst sollte eine solide Grundlage für die technologischen Implementationen geschaffen werden. Darunter fällt zum Beispiel die Flexibilisierung, Sicherung und Wartung qualitativer Wi-Fi Verbindungen, welche den Kern des intelligenten Arbeitsplatzes bilden. Außerdem sollte auch Unified Communication (UC)\nomenclature{UC}{Unified Communication} sichergestellt werden, da beispielsweise bei Videokonferenzen Räume mit Ausstattung die auf Internet of Things (IoT)\nomenclature{IoT}{Internet of Things} Technologie basieren alle Teilnehmer und deren Bedürfnisse automatisch erkennen können.\footcite[Vgl.][]{SCmagazineUK} UC oder auch Unified Communication \& Collaboration (UCC)\nomenclature{UCC}{Unified Communication \& Collaboration} vereinheitlicht mehrere Kommunikationswege und -Möglichkeiten zwischen Mitarbeitern aber auch Kunden, um den Zugriff auf alle Daten, Informationen und Geräte jederzeit und von jedem Ort zu gewährleisten.\footcite[Vgl.][]{CompWoche} IoT basierte Geräte sind dazu in der Lage, sich mikroprozessorgesteuert untereinander über ein digitales Netz zu unterhalten.\footcite[Vgl.][]{mittelstand}

Um eine Optimierung des Managements der Räumlchkeiten zu erreichen, muss die Abhängigkeit von manuellen Kontrollen reduziert werden. Dafür sind keine weiteren technologischen Systeme notwendig, dies kann durch bereits bestehende Funktionalitäten abgedeckt werden. Beispielsweise ließen sich Wi-Fi Nutzungsdaten dergestalt auswerten, dass die Bewegungen der Mitarbeiter verfolgt werden und in Kombination mit der Nutzung der per LAN angebundenen Geräte eine Wärmekarte produziert wird, um unbenutzte Räume zu identifizieren und dadruch an Heiz- und Stromkosten zu sparen. Außerdem unterstützt die Fähigkeit, Dienstleistungen zu kontrollieren und zu personalisieren, eine gute Infrastruktur. Schon kleine Anwendungen zur Kontrolle von Zugriff und Energieverbrauch können hierbei nützlich sein und lassen sich von jedermann bedienen. Der wichtigste Aspekt bei der Einführung eines Smart Workplace ist allerdings die Vereinheitlichung der Kommunikation einzelner Technologien. Da die verschiedenen Systeme unter Umständen nicht die gleiche Sprache sprechen, muss ein Middleware-System zur Übersetzung implementiert werden, welches anbieterunabhängig funktioniert und möglichst praktikabel skalierbar ist.
\footcite[Vgl.][]{SCmagazineUK}
