\newpage
\section{Rechtliche Grundlagen}
Vorgaben bezüglich der Gestaltung moderner Bildschirmarbeitsplätze sind in mehreren Gesetzen verankert. Schon 1996 wurde mit der BildschirmArbV (Bildschirmarbeitsverordnung)\nomenclature{BildschArbV}{Bildschirmarbeitsverordnung} allerdings eine speziell darauf bezogene Verordnung erlassen. Gemäß § 1 Abs. 2 gilt sie nicht für Bildschirmarbeitsplätze, die beweglich sind und nicht über einen längeren Zeitraum am gleichen Arbeitsplatz verwendet werden und für Geräte mit kleinem Bildschirm. Daher können Wearables trotz ihrer kleinen Bildschirme als primäre Arbeitsmittel genutzt werden.

Laptops hingegen können nur verwendet werden, wenn sich der Arbeitsplatz stetig wechselt, da sie in diesem Fall nicht von der BildschArbV betroffen sind. Im Anhang der BildschArbV werden die Anforderungen an Bildschirmgeräte, Tastaturen, sonstige Arbeitsmittel und die Arbeitsumgebung gestellt. Hier wird unter anderem in den Punkten 5-7 festgelegt, dass der Bildschirm frei dreh- und neigbar und die Tastatur vom Bildschirm getrennt sein muss. Da bei einem Laptop zumeist Bildschirm und Tastatur untrennbar voneinander sind, muss eine externe Tastatur am Laptop angeschlossen werden um eine Nutzung zu erlauben.

Die CGLX-Technologie kann für Meetings verwendet werden, da diese meist nicht lange dauern. Als fester Arbeitsplatz können sie allerdings nicht genutzt werden, weil kein ergonomisches Arbeiten mit einem Multi-Touch-Table und einer Leinwand möglich ist. Die Thin Clients der MOVE-Architektur müssen zur Verwendung die vorgeschriebene Ausstattung besitzen und die Bedingungen für eine gesunde Arbeitsumgebung wie eine Lichtschutzvorrichtung für die Fenster oder eine erträgliche Luftfeuchtigkeit müssen erfüllt sein, um das MOVE-System implementieren zu können.\nocite{BildschArbV}